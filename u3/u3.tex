\documentclass{article}
\usepackage{amsmath,amssymb,amsfonts,amsthm}
\usepackage[german]{babel}
\usepackage[utf8]{inputenc}
\usepackage{fancyhdr}
\usepackage{geometry}
\usepackage{tikz}
\usetikzlibrary{automata,positioning}
\geometry{top=20mm, left=20mm, right=10mm, bottom=15mm}


%\geometry{a4paper, top=25mm, left=40mm, right=10mm, bottom=30mm,headsep=10mm, footskip=12mm}


\newcommand{\hr}{\begin{center} \line(1,0){450} \end{center}}

\pagestyle{fancy}
\lhead{Christian Lengert}
\rhead{\today}
\chead{153767}
%\rfoot{Page \thepage}

\begin{document}

\begin{center}
	\section*{Konvexe Optimierung}
	\subsection*{2. Übungsserie}
\end{center}

\subsubsection*{Aufgabe 7}
\begin{center}
Ein Kegel $K \subset  \mathbb{R}^n$ ist konvex $\Leftrightarrow K + K \subseteq K$.
\end{center}


\begin{equation*}
\begin{split}
p,q \in K &\Leftrightarrow \big(k_1 \cdot p \in K\big) \land \big(k_2 \cdot q \in K\big) ,k_i > 0\\
&\Leftrightarrow \{ (1-t)\cdot k_1\cdot p + t\cdot k_2\cdot q \} \subseteq K,t \in \left[0,1 \right]\\
&\Leftrightarrow \{ p + q \} \subseteq K\\
&\Leftrightarrow K + K \subseteq K
\end{split}
\end{equation*}



\hr
\subsubsection*{Aufgabe 9}

	$$f: \mathbb{R}^n \rightarrow \mathbb{R}$$
	$$y \in \mathbb{R}^n$$
	$$f(x) = \vert\vert x -y \vert \vert$$

	\begin{center}
		$p,q \in \mathbb{R}^n,t \in \left[0,1 \right]$\\
		$f$ ist konvex $\Leftrightarrow f((1-t)p+tq) \leq (1-t)f(p)+ tf(q)$ 
	\end{center}
	
\begin{equation*}
\begin{split}
(1-t)f(p)+ tf(q) &=(1-t) \vert \vert p -y\vert \vert + t\vert \vert q -y\vert \vert\\
&=  \vert \vert (1-t) (p - y)\vert \vert + \vert \vert t(q -y)\vert \vert\\
&= \vert \vert  (1-t)p - (1-t)y\vert \vert + \vert \vert tq -ty\vert \vert \\
\end{split}
\end{equation*}


\begin{equation*}
\begin{split}
\vert \vert  (1-t)p - (1-t)y\vert \vert + \vert \vert tq -ty\vert \vert &\geq \vert \vert (1-t)p - (1-t)y +tq -ty \vert \vert \\
&= \vert \vert (1-t)p - y +ty +tq -ty\vert \vert \\
&= \vert \vert (1-t)p + tq - y\vert \vert\\
\end{split}
\end{equation*}


Somit ist $f$ konvex.


\hr
\subsubsection*{Aufgabe 11}
$$Q  \in R^{n\times n};q \in \mathbb{R}^n; c \in \mathbb{R}$$
$$\min\limits_{x \in \mathbb{R}^n} f(x) = \frac{1}{2} x^TQx+q^Tx+c$$


Es gilt:
\begin{equation}
\begin{split}
f((1-t)p + tq) = (1-t)f(p)+tf(q) -\frac{1}{2}t(1-t)(p-q)^TA(p-q)\\ 
\end{split}\label{pre}
\end{equation}



\begin{equation}
\begin{split}
&f((1-t)p+tq) \leq (1-t)f(p)+ tf(q)\\
\Leftrightarrow&~ 0 \leq (1-t)f(p)+ tf(q) - f((1-t)p+tq)\\
\Leftrightarrow&~ 0 \leq \frac{1}{2}t(1-t)(p-q)^TA(p-q)
\end{split}\ref{res}
\end{equation}


Sei $A \succeq 0$,dann gilt: $\frac{1}{2}t(1-t)(p-q)^TA(p-q) \geq 0 \Rightarrow f$ ist konvex.\\ 
Sei $A \succ 0$,dann gilt: $\frac{1}{2}t(1-t)(p-q)^TA(p-q) > 0 \Rightarrow f$ ist strikt konvex.


\paragraph{Zeige Gleichung \ref{pre}}


\begin{equation}
\begin{split}
f((1-t)a+tb) &= \frac{1}{2}((1-t)a+tb)^T Q ((1-t)a+tb)+ q^T((1-t)a+tb) + c \\
&= \frac{1}{2}((1-t)a^T + tb^T) ((1-t)Qa + Qtb) + q^T((1-t)a+tb) + c\\
&= \frac{1}{2}(1-t)^2 a^TQa + \frac{1}{2}t(1-t)a^TQb + \frac{1}{2}t(1-t)b^TQa + \frac{1}{2}t^2b^TQb + (1-t)q^Ta+t q^Tb +c%&= (1-t)\Big(\frac{1}{2} a^TQa+q^Ta+c\Big) + t\Big(\frac{1}{2} b^TQb+q^Tb+c\Big)
\end{split}
\end{equation}


\begin{equation}
\begin{split}
(1-t)f(a) + (t)f(b) &= (1-t)(\frac{1}{2} a^TQa+q^Ta+c) 									+ t(\frac{1}{2} b^TQb+q^Tb+c)\\
					&= \frac{1}{2} a^TQa-t\frac{1}{2} a^TQa+q^Ta-tc -tq^Ta+c + t\frac{1}{2} b^TQb+tq^Tb+tc\\
&=\frac{1}{2}(1-t)a^TQa +(1-t)q^Ta + t\frac{1}{2} b^TQb+tq^Tb +c
\end{split}
\end{equation}


\begin{equation}
\begin{split}
\frac{1}{2}(1-t)a^TQa +(1-t)q^Ta + t\frac{1}{2} b^TQb+tq^Tb +c \dots\\
\dots -\big(\frac{1}{2}(1-t)^2 a^TQa + \frac{1}{2}t(1-t)a^TQb + \frac{1}{2}t(1-t)b^TQa + \frac{1}{2}t^2b^TQb + (1-t)q^Ta+t q^Tb +c \big) = 0\\
\frac{1}{2}(1-t)a^TQa +(1-t)q^Ta + t\frac{1}{2} b^TQb+tq^Tb +c\dots\\
\dots -\frac{1}{2}(1-t)^2 a^TQa - \frac{1}{2}t(1-t)a^TQb - \frac{1}{2}t(1-t)b^TQa - \frac{1}{2}t^2b^TQb - (1-t)q^Ta-t q^Tb -c = 0\\
\frac{1}{2}(1-t)a^TQa -\frac{1}{2}(1-t)^2 a^TQa + t\frac{1}{2} b^TQb - \frac{1}{2}t^2b^TQb - t\frac{1}{2}(1-t)a^TQb - \frac{1}{2}t(1-t)b^TQa  = 0\\
\frac{1}{2}t(1-t)a^TQa + \frac{1}{2} t(1-t)b^TQb                  - t\frac{1}{2}(1-t)a^TQb - \frac{1}{2}t(1-t)b^TQa  = 0\\
a^TQa +  b^TQb                  - a^TQb - b^TQa  = 0\\
(a-b)^TQ(a-b)  = 0\\
\end{split}
\end{equation}


\end{document}
