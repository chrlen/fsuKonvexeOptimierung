\documentclass{article}
\usepackage{amsmath,amssymb,amsfonts,amsthm}
\usepackage[german]{babel}
\usepackage[utf8]{inputenc}
\usepackage{fancyhdr}
\usepackage{geometry}
\usepackage{multicol}
\usepackage{tikz}
\usetikzlibrary{automata,positioning}
\geometry{top=20mm, left=20mm, right=10mm, bottom=15mm}


%\geometry{a4paper, top=25mm, left=40mm, right=10mm, bottom=30mm,headsep=10mm, footskip=12mm}


\newcommand{\hr}{\begin{center} \line(1,0){450} \end{center}}
\newcommand{\R}{\mathbb R}
\newcommand{\tr}{^\mathsf{T}}
\newcommand{\F}{\mathcal{F}}
\newcommand{\norm}[1]{\left\lVert#1\right\rVert}
\DeclareMathOperator{\Id}{Id}
\DeclareMathOperator{\diag}{diag}
\DeclareMathOperator{\epi}{epi}
\DeclareMathOperator{\co}{co}
\DeclareMathOperator{\interior}{int}
\DeclareMathOperator{\Proj}{Pr}
\DeclareMathOperator*{\argmin}{argmin}
\DeclareMathOperator*{\argmax}{argmax}
\newcommand{\midd}{\mathrel{}\middle|\mathrel{}}




\pagestyle{fancy}
\lhead{Christian Lengert}
\rhead{\today}
\chead{153767}
%\rfoot{Page \thepage}

\begin{document}

\begin{center}
	\section*{Konvexe Optimierung}
	\subsection*{3. Übungsserie}
\end{center}

\subsubsection*{Aufgabe 18}
%\paragraph{(a)}
%\begin{equation*}
%    \begin{aligned}
%       \min\limits_{x \in \mathbb{R}^n} \quad \sum\limits_{i = 1}^m \left( x_1\xi_i + x_2 -\eta_i %\right)^2 = \min\limits_{x \in \mathbb{R}^n} \quad f(x)
%    \end{aligned}
%\end{equation*}
%\begin{equation*}
%    \begin{aligned}
%    f(x) &= \sum\limits_{i = 1}^m \left( x_1\xi_i + x_2 -\eta_i \right)^2
%    & 
%    \end{aligned}
%\end{equation*}

\paragraph{(b)}
\begin{equation*}
    f(x)=\frac{1}{2} \vert \vert Ax -b \vert \vert^2
\end{equation*}
\begin{equation}\label{df}
    \nabla_x f(x) = \frac{\partial}{\partial x}f(x)=A\tr \left(Ax -b\right)
\end{equation}
Berechne Nullstellen der ersten Ableitung:
\begin{align*}
    \eqref{df} \overset{!}{=} 0_m &\Rightarrow  A\tr \left(Ax -b\right)= 0_m\\
    &\Rightarrow A\tr Ax - A\tr b = 0_m\\
    &\Rightarrow A\tr Ax  = A\tr b\\
    &\Rightarrow x^*  = (A\tr A)^{-1}A\tr b\\
\end{align*}
Da $A$ den vollen Spaltenrang existiert das Inverse von $A\tr A$ und die Gleichung wird lösbar, wenn eine Schätzung von $b$ zur Verfügung steht.

\hr
\subsubsection*{Aufgabe 21}

\begin{equation*}
    \begin{split}
     \min\limits_{x \in \mathbb{R}^n} &\quad f(x) = (x_1 - 3)^2 + (x_2 -4)^2\\
    \text{s.t.} &\quad \quad 2x_1 + x_2 \leq 6  
\end{split}
\end{equation*}

\paragraph{Konvexität}
Das Problem ist konvex, wenn die Menge der zulässigen Punkte $\mathcal{C}$ konvex ist und die Funktion $f$ konvex auf dieser Menge ist.
\begin{align*}
    f(x) &= \left(x_1 - 3\right)^2 + \left(x_2 -4\right)^2\\
    &= \begin{pmatrix}x_1-3\\x_2 -4\end{pmatrix}\tr \begin{pmatrix}x_1-3\\x_2 -4\end{pmatrix}\\
    &= \begin{pmatrix}x_1-3\\x_2 -4\end{pmatrix}\tr \begin{pmatrix}1&0\\0&1\end{pmatrix}  \begin{pmatrix}x_1-3\\x_2 -4\end{pmatrix} + 0_2 r + 0_2, \quad r \in \R^2
\end{align*}
Die Funktion $f$ hat quadratische Form und ist somit konvex.





%Sei $\mathcal{C} = \{ x \in \R^2 \mid 2x_1 +x_2 \leq 6 \} = \{ x \in \R^2 \mid \left( x\right)\tr \left(2,1\right) \leq 6 \} $ und  $a,b \in \mathcal{C}$:



%\begin{align}
%    \left((1-t)a + tb\right)\tr \left(2,1\right) \leq 6
%\end{align}

\paragraph{Lösung}
Stelle Lagrange-Multiplikator für Nebenbedingung auf:
\begin{equation*}
    \mathcal{L}(x_1,x_2,\mu) = (x_1 - 3)^2 + (x_2 -4)^2 + \mu(2x_1 + x_2 - 6 )
\end{equation*}
Bilde partielle Ableitungen und setze gleich Null:

\begin{equation}\label{dx1}
\begin{aligned}
            &\nabla_{x_1}\mathcal{L}(x_1,x_2,\mu) = 2(x_1-3)+\mu \overset{!}{=} 0 \\
        \Rightarrow& x_1 = \frac{1}{2}(6-\mu)
\end{aligned}
\end{equation}

\begin{equation}\label{dx2}
\begin{aligned}
    &\nabla_{x_2}\mathcal{L}(x_1,x_2,\mu) = 2(x_2-4)+\mu \overset{!}{=} 0 \\
    \Rightarrow& x_2 = \frac{1}{2}(8-\mu)
\end{aligned}
\end{equation}

\begin{equation}\label{dmu}
\begin{aligned}
    & \nabla_{\mu}\mathcal{L}(x_1,x_2,\mu) =2x_1+x_2-6 \overset{!}{=} 0 \\
& \Rightarrow 2x_1 = 6 - x_2
\end{aligned}
\end{equation}

Löse das sich ergebende Gleichungssystem:
\begin{equation}\label{mueq}
\begin{aligned}
    \eqref{dx1},\eqref{dx2},\eqref{dmu} &\Rightarrow 2\left( \frac{1}{2} \left(6 - \mu \right) \right) = 6 - \frac{1}{2} \left( 8-\mu \right)\\
    \Rightarrow & 6 - \mu  = 6 - \frac{1}{2} \left( 8-\mu \right)\\
    \Rightarrow &  - \mu  =  - \frac{1}{2} \left( 8-\mu \right)\\
    \Rightarrow &   \mu  =   \frac{1}{2} \left( 8-\mu \right)  \\
    \Rightarrow &   \frac{3}{2}\mu  =  4 \\
    \Rightarrow &   \mu  =  \frac{8}{3} \\
\end{aligned}
\end{equation}
\begin{equation}\label{x1eq}
\begin{aligned}
    \eqref{dx1},\eqref{mueq} \Rightarrow& x_1 = \frac{1}{2}\left(6-\frac{8}{3}\right)\\
    \Rightarrow & x_1 = \left(3-\frac{1 \cdot 8}{2\cdot 3}\right) = \frac{18}{6} -  \frac{8}{6} = \frac{10}{6} = \frac{5}{3}\\
\end{aligned}
\end{equation}
\begin{equation}\label{x2eq}
\begin{aligned}
    \eqref{dx2},\eqref{mueq} \Rightarrow& x_2 = \frac{1}{2}\left(8-\frac{8}{3}\right)\\
    \Rightarrow & x_2 = \left(4-\frac{1 \cdot 8}{2\cdot 3}\right) = \frac{24}{6} -  \frac{8}{6} = \frac{16}{6} = \frac{8}{3}\\
\end{aligned}
\end{equation}

Durch Einsetzen ergibt sich der Funktionswert:
\begin{equation}
\begin{aligned}
    \mathcal{L}\left(\frac{5}{3},\frac{8}{3},\frac{8}{3}\right) &=  \left(\left(\left(\frac{5}{3} - 3\right)\right)^2 + \left(\left(\frac{8}{3} -4\right)\right)\right)^2 + \frac{8}{3}\left(\left(2*\frac{5}{3} + \frac{8}{3} - 6 \right)\right)\\
    &=  \frac{16}{9} + \frac{16}{9} + \frac{8}{3} \left( \frac{18}{3} - \frac{18}{3} \right)\\
     &=  \frac{32}{9}
\end{aligned}
\end{equation}



\end{document}
