\documentclass{article}
\usepackage{amsmath,amssymb,amsfonts,amsthm}
\usepackage[german]{babel}
\usepackage[utf8]{inputenc}
\usepackage{fancyhdr}
\usepackage{geometry}
\usepackage{tikz}
\usetikzlibrary{automata,positioning}
\geometry{top=20mm, left=20mm, right=10mm, bottom=15mm}


\newcommand{\hr}{\begin{center} \line(1,0){450} \end{center}}
\newcommand{\R}{\mathbb R}
\newcommand{\tr}{^\mathsf{T}}
\newcommand{\F}{\mathcal{F}}
\newcommand{\norm}[1]{\left\lVert#1\right\rVert}
\DeclareMathOperator{\Id}{Id}
\DeclareMathOperator{\diag}{diag}
\DeclareMathOperator{\epi}{epi}
\DeclareMathOperator{\co}{co}
\DeclareMathOperator{\interior}{int}
\DeclareMathOperator{\Proj}{Pr}
\DeclareMathOperator*{\argmin}{argmin}
\DeclareMathOperator*{\argmax}{argmax}
\newcommand{\midd}{\mathrel{}\middle|\mathrel{}}


\pagestyle{fancy}
\lhead{Christian Lengert}
\rhead{\today}
\chead{153767}
%\rfoot{Page \thepage}

\begin{document}

\begin{center}
	\section*{Konvexe Optimierung}
	\subsection*{5. Übungsserie}
\end{center}

\subsubsection*{Aufgabe 29}
\begin{equation}
f_1(x) = \sum\limits_{i=1}^n x_i = \norm{x}^2
\end{equation}
Zu zeigen: $f_1$ ist gleichmäßig konvex mit Konvexitätsparameter $2$. $f_1$ ist gleichmäßig konvex wenn gilt:
\begin{equation}
(1-t)\norm{x}^2 + t \norm{x}^2 \geq \norm{(1-t)x + ty}^2 + t(1-t)\mu \norm{x -y}^2
\end{equation}
\end{document}
