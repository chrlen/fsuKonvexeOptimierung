\documentclass{article}
\usepackage{amsmath,amssymb,amsfonts,amsthm}
\usepackage[german]{babel}
\usepackage[utf8]{inputenc}
\usepackage{fancyhdr}
\usepackage{geometry}
\usepackage{tikz}
\usepackage{xcolor}
\usetikzlibrary{automata,positioning}
\geometry{top=20mm, left=20mm, right=10mm, bottom=15mm}


\newcommand{\hr}{\begin{center} \line(1,0){450} \end{center}}
\newcommand{\R}{\mathbb R}
\newcommand{\tr}{^\mathsf{T}}
\newcommand{\F}{\mathcal{F}}
\newcommand{\norm}[1]{\left\lVert#1\right\rVert}
\DeclareMathOperator{\Id}{Id}
\DeclareMathOperator{\diag}{diag}
\DeclareMathOperator{\epi}{epi}
\DeclareMathOperator{\co}{co}
\DeclareMathOperator{\interior}{int}
\DeclareMathOperator{\Proj}{Pr}
\DeclareMathOperator*{\argmin}{argmin}
\DeclareMathOperator*{\argmax}{argmax}
\newcommand{\midd}{\mathrel{}\middle|\mathrel{}}


\pagestyle{fancy}
\lhead{Christian Lengert}
\rhead{\today}
\chead{153767}
%\rfoot{Page \thepage}

\begin{document}

\begin{center}
	\section*{Konvexe Optimierung}
	\subsection*{5. Übungsserie}
\end{center}

\subsubsection*{Aufgabe 33}

 \begin{align*}
 h_\delta(x,y) 	&= \sum\limits_i \delta^2 \left( \sqrt{1+\left( \frac { \xi_i x_0 + x_1 -\eta_i}{\delta} \right)^2} -1 \right)\\
&= \sum\limits_i \delta^2 \left( \sqrt{1+ \frac {\xi_i^2 x_0^2 + 2 \xi_i x_0 x_1 - \xi_i x_0 \eta_i + x_1^2 -x_1 \eta_i -\xi_i x_0 \eta_i - \eta_i x_1 + \eta_i^2}{\delta^2}} -1 \right)\\
\end{align*}
\end{document}
